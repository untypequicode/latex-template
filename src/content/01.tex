\section{Tableaux et figures}

Cette section démontre comment créer et utiliser des tableaux et des figures dans votre document LaTeX.

\subsection{Tableaux simples}

Les tableaux sont créés avec l'environnement \texttt{table} et \texttt{tabular}. Voici un exemple :

\begin{table}[H]
	\centering
	\begin{tabular}{|l|c|r|}
		\hline
		\textbf{Colonne gauche} & \textbf{Colonne centre} & \textbf{Colonne droite} \\
		\hline
		Texte aligné à gauche   & Texte centré            & Texte aligné à droite   \\
		Ligne 2                 & 42                      & 3.14                    \\
		Ligne 3                 & 100                     & 2.71                    \\
		\hline
	\end{tabular}
	\caption{Exemple de tableau simple avec différents alignements}
	\label{tab:exemple_simple}
\end{table}

\subsection{Tableaux avancés}

Pour des tableaux plus complexes, vous pouvez créer des structures plus élaborées :

\begin{table}[H]
	\centering
	\begin{tabular}{|l|c|c|c|}
		\hline
		\textbf{Fonctionnalité} & \textbf{Version 1.0} & \textbf{Version 2.0} & \textbf{Version 3.0} \\
		\hline
		Bibliographie           & Non                  & Oui                  & Oui                  \\
		Images                  & Oui                  & Oui                  & Oui                  \\
		Tableaux                & Basique              & Basique              & Avancé               \\
		Équations               & Oui                  & Oui                  & Oui                  \\
		\hline
	\end{tabular}
	\caption{Comparaison des versions du template}
	\label{tab:comparaison}
\end{table}

\subsection{Insertion d'images}

Les images sont insérées avec l'environnement \texttt{figure} et la commande \texttt{\textbackslash includegraphics}.

\begin{figure}[H]
	\centering
	\IfFileExists{img/logo.png}{%
		\includegraphics[width=0.3\textwidth]{img/logo.png}
	}{%
		\fbox{\begin{minipage}[c][5cm][c]{0.6\textwidth}
				\centering
				\textbf{Logo non trouvé}\\[1em]
				\small Placez votre logo dans \texttt{src/img/logo.png}
			\end{minipage}}
	}
	\caption{Exemple d'insertion d'image : logo du projet}
	\label{fig:exemple}
\end{figure}

\textbf{Code utilisé :}
\begin{verbatim}
\begin{figure}[H]
  \centering
  \includegraphics[width=0.3\textwidth]{img/logo.png}
  \caption{Exemple d'insertion d'image}
  \label{fig:exemple}
\end{figure}
\end{verbatim}

\textbf{Astuces :}
\begin{itemize}
	\item Utilisez l'option \texttt{[H]} pour forcer le positionnement exact
	\item Contrôlez la taille avec \texttt{width=0.3\textbackslash textwidth} ou \texttt{scale=0.5}
	\item Formats supportés : PNG, JPG, PDF
	\item Placez vos images dans le dossier \texttt{src/img/}
\end{itemize}
