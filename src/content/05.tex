\section{Gestion de la bibliographie}

Cette section explique comment utiliser le système de bibliographie intégré au template.

\subsection{Configuration}

Le template utilise \texttt{biblatex} avec le backend \texttt{biber} pour gérer les références bibliographiques de manière moderne et flexible.

La configuration se trouve dans \texttt{preamble.tex} :
\begin{verbatim}
\usepackage[
  backend=biber,
  style=numeric,
  sorting=nty,
  citestyle=numeric
]{biblatex}
\end{verbatim}

\subsection{Fichier de références}

Toutes les références sont stockées dans le fichier \texttt{references.bib}. Le fichier par défaut peut être personnalisé dans \texttt{env.sty} :

\begin{verbatim}
\renewcommand{\envBibFile}{mes-references.bib}
\end{verbatim}

\subsection{Types de citations}

Différents formats de citation sont disponibles :

\begin{description}
	\item[Citation simple] Selon \cite{example_article}, cette approche est efficace.
	\item[Citations multiples] Plusieurs études \cite{example_book,example_conference} le confirment.
	\item[Citation avec page] Comme mentionné dans \cite[p.~42]{example_book}.
	\item[Citation entre parenthèses] Cette méthode est reconnue \parencite{example_article}.
\end{description}

\subsection{Types de références supportés}

Le fichier \texttt{references.bib} supporte de nombreux types d'entrées :

\begin{itemize}
	\item \textbf{Articles scientifiques} (\texttt{@article}) : \cite{example_article}
	\item \textbf{Livres} (\texttt{@book}) : \cite{example_book}
	\item \textbf{Chapitres de livres} (\texttt{@incollection}) : \cite{example_chapter}
	\item \textbf{Sites web} (\texttt{@online}) : \cite{example_website}
	\item \textbf{Conférences} (\texttt{@inproceedings}) : \cite{example_conference}
	\item \textbf{Thèses} (\texttt{@phdthesis}) : \cite{example_thesis}
\end{itemize}

\subsection{Exemple d'entrée BibTeX}

Voici un exemple de référence dans \texttt{references.bib} :

\begin{verbatim}
@article{mon_article,
  author  = {Nom, Prénom and Autre, Auteur},
  title   = {Titre de l'article},
  journal = {Nom du Journal},
  year    = {2024},
  volume  = {42},
  pages   = {123--145},
  doi     = {10.1234/example.doi}
}
\end{verbatim}

\subsection{Compilation}

Pour que la bibliographie soit correctement générée, utilisez :

\begin{verbatim}
just build
\end{verbatim}

Ou manuellement :
\begin{verbatim}
pdflatex main.tex
biber main
pdflatex main.tex
pdflatex main.tex
\end{verbatim}

\subsection{Personnalisation du style}

Vous pouvez modifier le style de citation dans \texttt{preamble.tex}. Quelques options populaires :

\begin{itemize}
	\item \texttt{style=numeric} : Citations numérotées [1, 2, 3]
	\item \texttt{style=alphabetic} : Citations alphabétiques [Doe24]
	\item \texttt{style=authoryear} : Citations auteur-année (Doe, 2024)
	\item \texttt{style=apa} : Style APA
\end{itemize}

La bibliographie complète apparaît automatiquement en fin de document.
