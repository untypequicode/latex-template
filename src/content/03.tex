\section{Équations mathématiques}

Cette section démontre comment écrire des équations mathématiques en LaTeX.

\subsection{Formules en ligne}

Les formules mathématiques peuvent être insérées dans le texte : $E = mc^2$ ou $a^2 + b^2 = c^2$.

Pour cela, utilisez les symboles \texttt{\$} : \verb|$E = mc^2$|

\subsection{Équations numérotées}

Les équations importantes peuvent être mises en évidence et numérotées :

\begin{equation}
	\int_{-\infty}^{+\infty} e^{-x^2} dx = \sqrt{\pi}
	\label{eq:gaussienne}
\end{equation}

\begin{equation}
	f(x) = \sum_{n=0}^{\infty} \frac{f^{(n)}(a)}{n!}(x-a)^n
	\label{eq:taylor}
\end{equation}

\subsection{Équations non numérotées}

Pour des équations sans numérotation, utilisez l'environnement avec étoile :

\begin{equation*}
	\nabla \cdot \mathbf{E} = \frac{\rho}{\epsilon_0}
\end{equation*}

\subsection{Systèmes d'équations}

L'environnement \texttt{align} permet d'aligner plusieurs équations :

\begin{align}
	x + y  & = 10 \label{eq:sys1} \\
	2x - y & = 5 \label{eq:sys2}
\end{align}

\subsection{Matrices}

Les matrices utilisent l'environnement \texttt{pmatrix}, \texttt{bmatrix}, ou \texttt{vmatrix} :

\begin{equation}
	A = \begin{pmatrix}
		a_{11} & a_{12} & a_{13} \\
		a_{21} & a_{22} & a_{23} \\
		a_{31} & a_{32} & a_{33}
	\end{pmatrix}
\end{equation}

\subsection{Symboles mathématiques courants}

Voici quelques symboles fréquemment utilisés :

\begin{itemize}
	\item Lettres grecques : $\alpha, \beta, \gamma, \Delta, \Omega$
	\item Opérateurs : $\sum, \prod, \int, \partial, \nabla$
	\item Relations : $\leq, \geq, \neq, \approx, \equiv$
	\item Ensembles : $\mathbb{N}, \mathbb{Z}, \mathbb{Q}, \mathbb{R}, \mathbb{C}$
	\item Logique : $\forall, \exists, \in, \notin, \subset$
	\item Flèches : $\rightarrow, \Rightarrow, \leftrightarrow, \Leftrightarrow$
\end{itemize}
