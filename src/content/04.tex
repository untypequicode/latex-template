\section{Références croisées et liens}

Cette section montre comment créer des références internes et des liens hypertextes dans votre document.

\subsection{Références à des sections}

Vous pouvez référencer d'autres sections du document. Par exemple, voir la Section \ref{sec:labels} pour plus de détails sur les labels.

Les tableaux et figures peuvent aussi être référencés : le Tableau \ref{tab:exemple_simple} de la première section montre un exemple simple.

\subsection{Utilisation des labels}
\label{sec:labels}

Pour créer une référence, il faut :

\begin{enumerate}
	\item Placer un \texttt{\textbackslash label\{nom\_unique\}} après l'élément
	\item Utiliser \texttt{\textbackslash ref\{nom\_unique\}} pour y faire référence
\end{enumerate}

\textbf{Exemple de code :}
\begin{verbatim}
\section{Ma section}
\label{sec:ma_section}

Voir la Section \ref{sec:ma_section}.
\end{verbatim}

\subsection{Références aux équations}

Les équations peuvent être référencées de la même manière. Par exemple, l'équation de la gaussienne \eqref{eq:gaussienne} et la série de Taylor \eqref{eq:taylor} ont été présentées dans la section précédente.

Utilisez \texttt{\textbackslash eqref\{\}} pour obtenir automatiquement les parenthèses.

\subsection{Liens hypertextes}

Le package \texttt{hyperref} permet de créer des liens cliquables :

\begin{itemize}
	\item Lien externe : \href{https://www.latex-project.org/}{Site officiel de LaTeX}
	\item Email : \href{mailto:exemple@example.com}{exemple@example.com}
	\item URL simple : \url{https://github.com}
\end{itemize}

\textbf{Code exemple :}
\begin{verbatim}
\href{https://example.com}{Texte du lien}
\url{https://example.com}
\end{verbatim}

\subsection{Table des matières}

La table des matières est générée automatiquement avec \texttt{\textbackslash tableofcontents} et inclut toutes les sections, sous-sections et sous-sous-sections.

\subsection{Notes de bas de page}

Vous pouvez ajouter des notes de bas de page\footnote{Ceci est un exemple de note de bas de page} avec la commande \texttt{\textbackslash footnote\{\}}.
