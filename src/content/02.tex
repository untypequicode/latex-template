\section{Listes et mise en forme du texte}

Cette section présente les différents types de listes et options de mise en forme du texte.

\subsection{Listes à puces}

Les listes à puces utilisent l'environnement \texttt{itemize} :

\begin{itemize}
	\item Premier élément de la liste
	\item Deuxième élément
	\item Troisième élément avec des sous-éléments :
	      \begin{itemize}
		      \item Sous-élément 1
		      \item Sous-élément 2
	      \end{itemize}
	\item Dernier élément
\end{itemize}

\subsection{Listes numérotées}

Les listes numérotées utilisent l'environnement \texttt{enumerate} :

\begin{enumerate}
	\item Première étape du processus
	\item Deuxième étape avec détails :
	      \begin{enumerate}
		      \item Sous-étape A
		      \item Sous-étape B
		      \item Sous-étape C
	      \end{enumerate}
	\item Troisième étape
	\item Quatrième et dernière étape
\end{enumerate}

\subsection{Listes de descriptions}

Les listes de descriptions utilisent l'environnement \texttt{description} :

\begin{description}
	\item[LaTeX] Système de composition de documents de haute qualité
	\item[BibTeX/Biber] Outils de gestion bibliographique
	\item[Just] Gestionnaire de commandes moderne
	\item[Nix] Gestionnaire de paquets reproductible
\end{description}

\subsection{Mise en forme du texte}

Différentes options de formatage sont disponibles :

\begin{itemize}
	\item \textbf{Texte en gras} avec \texttt{\textbackslash textbf\{\}}
	\item \textit{Texte en italique} avec \texttt{\textbackslash textit\{\}}
	\item \texttt{Police à chasse fixe (code)} avec \texttt{\textbackslash texttt\{\}}
	\item \underline{Texte souligné} avec \texttt{\textbackslash underline\{\}}
	\item {\large Texte agrandi} avec \texttt{\{\textbackslash large ...\}}
	\item {\small Texte réduit} avec \texttt{\{\textbackslash small ...\}}
\end{itemize}

\subsection{Blocs de code}

Pour afficher du code source, utilisez l'environnement \texttt{verbatim} :

\begin{verbatim}
def hello_world():
    print("Hello, World!")
    return True
\end{verbatim}

Ou pour du code en ligne : \verb|x = y + z| avec \texttt{\textbackslash verb}.
