% Packages essentiels
\usepackage{amsmath}
\usepackage{amsfonts, amssymb}
\usepackage[french]{babel}
\usepackage[utf8]{inputenc}
\usepackage[T1]{fontenc}
\usepackage{fancyhdr}
\usepackage{geometry}
\usepackage{hyperref}
\usepackage{setspace}
\usepackage{graphicx}
\usepackage{float}
\usepackage{titlesec}

% Masquer les numéros de sections dans le document (mais les garder dans la table des matières)
\titleformat{\section}[hang]{\normalfont\Large\bfseries}{}{0pt}{}
\titleformat{\subsection}[hang]{\normalfont\large\bfseries}{}{0pt}{}
\titleformat{\subsubsection}[hang]{\normalfont\normalsize\bfseries}{}{0pt}{}
\titleformat{\paragraph}[runin]{\normalfont\normalsize\bfseries}{}{0pt}{}
\titleformat{\subparagraph}[runin]{\normalfont\normalsize\bfseries}{}{0pt}{}

% Charger la configuration du projet (env.sty)
\IfFileExists{env.sty}{%
  \usepackage{env}%
}{}

% Appliquer les chemins de recherche pour les images
\makeatletter
\@ifundefined{envGraphicPaths}{%
  \graphicspath{{img/}}%
}{%
  \graphicspath\envGraphicPaths%
}
\makeatother
